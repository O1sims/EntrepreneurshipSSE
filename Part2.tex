\part{Entrepreneurship in a Platonian Economy} 
\label{part:entrepreneurshipPlatonianEconomy}

\section*{Overview of Part II}

When discussing the representative states of the socio-economic space we defined the Platonian economy as an organic state within which the generation of wealth derives from social organisation founded on the horizontal division of labour. There does not exist hierarchical production organisations, firms, or platforms that generate economic wealth. Instead, wealth is generated through the interaction of specialised consumer-producers within a network of mutually beneficial bilateral exchanges. Economic agents are assumed to be equal within the trade such that there exists no exercise of authority within the exchange. The Platonian economy is, in a sense, the most simplistic social economy: there exists a juvenile, yet non-trivial, governance system and a set of economic interactions between individual consumer-producers, but no social organisation beyond this point. As a result the number of socio-economic roles are limited, the interaction inefficiency between agents is assumed to be large, and role-building costs are large.

Part~\ref{part:entrepreneurshipPlatonianEconomy} focuses on economic interaction and entrepreneurship within a Platonian socio-economic space. To do so we build on materials developed in Chapters~\ref{ch:relationalperspective},~\ref{ch:relationaltheory}, and~\ref{ch:entrepreneurship}. Focus is placed on the interaction between entrepreneurs, the development of new socio-economic roles, and the impact that entrepreneurship has on the interaction infrastructure of the socio-economic space and the positional attributes of individual consumer-producers. Lemma~\ref{con:positionalattributes}, Conjecture~\ref{conjecture:UniquePositions} and the overall discussion developed in Chapter~\ref{ch:entrepreneurship} play an important role throughout. Above, we noted that an entrepreneur can manipulate different layers of the socio-economic space: focus here is on identifying a relationship between entrepreneurship, unique relational positions of entrepreneurial agents within an interaction infrastructure, and how entrepreneurs attain and exploit power within a Platonian economy.

\subsection*{Chapter breakdown}

To express unique positions and power within a networked interaction infrastructure we extend the mathematical definition of the socio-economic space to include network theory as initially discussed in Section~\ref{sec:socialeconomicnetworks}. Chapter~\ref{ch:criticalnodes}, \emph{Middlemen as entrepreneurs}, introduces these theoretical network concepts as well as a set of metrics to measure the positional power of economic agents within a Platonian socio-economic space. The metrics are defined over critical nodes; which we claim represent entrepreneurs that have developed a unique socio-economic role and thus position within the interaction infrastructure. These metrics are applied to the empirical application of the elite Florentine families introduced in Section~\ref{sec:HouseofMedici}.

Chapter~\ref{ch:blocks}, \emph{The formation of extractive structures in networks}, directly extends the notions discussed in Chapter~\ref{ch:criticalnodes} to multiple economic agents and thus sets of nodes. Non-cooperative game theory the main methodological tool used to assess the individualistic cooperation of economic agents within a networked environment. An important aspect of entrepreneurship is the alteration of the interaction infrastructure of the socio-economic space. The formation of `blocks' within an interaction infrastructure is a direct representation of network-oriented entrepreneurial action; the outcome of which is the formation of new socio-economic roles. Furthermore, this form of entrepreneurial action can be viewed in terms of merger, cartel, and acquisition activity.