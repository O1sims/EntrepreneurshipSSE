\part{A Theory of Entrepreneurship} 
\label{part:generalTheoryEntrepreneurship}

\section*{Overview of Part I}

The entrepreneur has remained an elusive character in economic theory. A reason for this derives from the modelling constraints informed by the traditional economic methodology which revolves around methodological individualism, instrumentalism and equilibration. To inform a holistic view of the entrepreneur we provide a novel basis within which to integrate the notion. Part~\ref{part:generalTheoryEntrepreneurship} of this monograph provides a set of fundamental notions that define the relational perspective; a proposed theory in which to investigate the formation of functional social interactions, economic relationships and the development of a social division of labour based on an evolutionary perspective of human nature and sociality. From this basis, we describe the notion of the entrepreneur within the relational perspective. The entrepreneur---and entrepreneurial activity that they engage in---is fully developed within this framework and remains the main focus of this monograph. Therefore, the main purpose of Part~\ref{part:generalTheoryEntrepreneurship} is to introduce the relational perspective and the notion of the entrepreneur; this allows for a more formal discussion of the entrepreneur and entrepreneurial activities in Parts~\ref{part:entrepreneurshipPlatonianEconomy} and~\ref{part:entrepreneurshipPlatformEconomy} later in the monograph.

\subsection*{Chapter breakdown}

The development of the relational perspective and the introduction of the entrepreneur is expressed over three chapters. Chapter~\ref{ch:relationalperspective}, \emph{Toward a relational perspective}, discusses the underlying axioms and hypotheses at the basis of the relational perspective. We discuss the evolutionary characteristics of social actors and how these characteristics inform our perception of the economic agent. An economic agent is considered as socially dependent and whose actions and specialisations are informed from the rules and behaviours of society as a whole. Furthermore, and deviating from the traditional theory of the economic agent, we characterise the agent with both productive and consumptive attributes; as such we interchange between the terminology of an economic agent and a consumer-producer.

This chapter provides an insight into the notion of socio-economic roles; an economic agents' reflection of the environment in which they interact, in the institutions of society and the division of labour that exists within the economy. Socio-economic roles are important for economic interaction and their creation and development is important for discussing the entrepreneur and the entrepreneurial function. Specifically, all economic interaction revolves around the existence of well-defined socio-economic roles, this interaction given the presence of institutions is discussed.

Chapter~\ref{ch:relationaltheory}, \emph{Growth and development of the socio-economic space}, provides an overview of the socio-economic space, an environmental notion that contains economic agents, the interaction infrastructure that they develop through the formation of economic interactions and relationships and the institutions and systems of governance that guide interaction between consumer-producers. The notion of a socio-economic space articulates how the fundamental notions of the relational perspective interact. From this a debate of the evolution of the socio-economic space is provided; first with a discussion of the theoretical and realised states of the socio-economic space, and then with a more general discussion of its growth and development. This discussion is complemented with the notion of the entrepreneur and the entrepreneurial function.

After developing the theoretical foundation of the relational perspective, Chapter~\ref{ch:entrepreneurship}, \emph{Entrepreneurship and the entrepreneurial function}, provides an in-depth discussion of the entrepreneur and the entrepreneurial function. Specific interest is placed on the interaction between entrepreneurship, the governance system of the socio-economic space and the positions of entrepreneurs within a matrix of interactions. We link the notion of the entrepreneur and the actions of entrepreneurship to the deepening of the division of labour and the provision of more diverse socio-economic roles. An application is given in the form of the House of Medici; we illustrate how, through the formation of a new socio-economic role the Medici achieved a powerful position within society. This example illustrates not only the impact of entrepreneurship and the entrepreneurial function but also provides a good example of the elements of the relational perspective.
